% !TeX root = 00_main.tex

\chapter[Related Work]{Related Work}
\label{sec:rw}
This section provides a survey of the state-of-the-art in spatio-temporal user identification, trajectory based user linkage, and trajectory privacy. User identification is focused on identifying the same user again within the same database, while user linkage is focused on linking two users together across multiple sources of data. This work assumes that user trajectory data is fully available, without any notion of privacy preservation. This assumption is appropriate in the experimental evaluation, using publicly available Twitter data. However, other trajectory data sets may employ some form of privacy, thus it is important to understand privacy methods, which might be used on trajectory data.

\section{User Identification}
A problem similar to the problem of trajectory based user-identification was considered in \citeauthor{DeMontjoye2013} \cite{DeMontjoye2013}. This work estimates the number of points needed to uniquely identify an individual trajectory. The dataset contained 15 months of data on ~1.5M users in a small European country. Each time a user connected to a mobile phone tower to send or receive a call or text message, a tower location and time, with a resolution of one hour, was recorded. There are almost 6500 unique antennas in the dataset, and on average each user has 114 interactions per month. Among this dataset, they found that four randomly chosen points of a trajectory were enough to uniquely identify 95\% of the trajectories, and two randomly chosen points were enough to identify 50\% of the trajectories. However, the question whether a trajectory is unique, is different to the problem of user-identification tackled in this work.

The user-identification method in \citeauthor{DeMontjoye2013} assumes that a trajectory of the user to be identified is already in the database. Thus, a new trajectory, which has not been seen before, cannot be classified. Summarizing, the work in \citeauthor{DeMontjoye2013}, aims at identifying individual trajectories, rather than individual users. Their work provided an initial framework to build this work on.

The work presented in \citeauthor{Bettini2005} \cite{Bettini2005} investigates the problem of how to prevent the identification of actual persons behind the users of location based services. Therefore, the authors employ so-called location-based quasi identifiers, which are formed from historical spatio-temporal movement patterns that are gathered from location-based service requests as a potential privacy concern. However, the stated problem is slightly different from this work, as they make use of external sources to finally get the real names behind the pseudonames.

\section{User Linkage}
There are a variety of publications considering the problem of user linkage or more general record linkage. In the database community, record linkage generally aims at detecting duplicate records within one or several databases. Records describing the same entity may not share a common key or contain faulty attribute values, which makes the detection of such duplicates non-trivial. A survey on the proposed approaches can be found in \citeauthor{Elmagarmid2007} \cite{Elmagarmid2007}.

Considering networks, record linkage is widely understood as user linkage and is stated as the problem of linking corresponding identities from different communities appearing within one or many networks \citeauthor{Zafarani2009} \cite{Zafarani2009}. It is specifically tailored to the requirements of user identification in heterogeneous data by considering co-occurrences adjusted with a stimulus signal. The stimulus signal is derived from locations with frequent co-occurrences and decays with increasing distance to a trajectory. The stimulus signal allows this method to weight important locations, which helps to distinguish two users with very similar trajectories.

An important area of user linkage is social networks where the user linking problem aims at connecting user profiles from different platforms that are used by the same persons. \citeauthor{Liu2013} \cite{Liu2013} differentiate between three types of user linkage across social networks: user-profile-based methods, which use information provided by the profiles to connect corresponding profiles \cite{Malhotra2012}, user-generated-content-based approaches, which analyze the content published by the users to link profiles \cite{Liu2013} and user-behavior-model-based methods that generate models based on the (temporal) user behaviors and finally link user profiles based on the similarity of these models \cite{Liu2014}.

Most related to this approach is the recent work of \citeauthor{Cao2016} \cite{Cao2016}. In this work, the authors use various sources for the trajectories and propose a MapReduce-based framework called Automatic User Identification (AUI). They identify sample rate, temporal and spatial sparsity and the fact that people with a close relationship provide similar trajectories as distinct features of the data. Sparsity of the data is corrected by using a long time frame. Signal Based Similarity (SIG) is introduced as a measurement of the similarity of two trajectories. In contrast to that approach, this work uses sparser trajectories. While the authors of \citeauthor{Cao2016} do consider sparse social media data, they accumulate these trajectories during a long time interval of at least multiple months. In this work, a long term mobility history of user is not assumed to be  available. Instead, it aims at identifying users with as few observations as possible.

\section{Trajectory Privacy}
The predominantly used measurement for privacy is k-anonymity \cite{Sweeney2002}, which works with a closed world assumption and assures that, for each query that could be used to identify the identity of a user, at least $k-1$ other users are returned as possible results.

Common approaches to guarantee a defined degree of anonymity are suppression, obfuscation and generalization \cite{Hashem2007}. To achieve k-anonymity by suppression, every element that does not fit into an anonymity set is removed \cite{Byun2007, Lefevre2006}. For trajectories, oppression would require discarding observations in discriminative locations such as a user's home. While this method is effective, the use of only suppression can lead to a significant loss of information. Perturbation is another method used to obfuscate the data \cite{Aggarwal2004}. The goal is to generate a synthetic dataset with the same properties of the original dataset using a generative model. For generalization, $k$-groups of users could simply be unified into a single entity.

This work does not try to maintain privacy of users, and can be seen as an adversary approach of trying to breach the privacy of users. A highly relevant future piece of work is to investigate how existing privacy preservation methods for trajectories can be employed to suppress, obfuscate and generalize trajectories to minimize the user identification accuracy of this solutions, while further minimizing the loss of information in the data.

A more refined version of k-anonymity is l-diversity, which addresses some shortcomings of k-anonymity \cite{Machanavajjhala2006}, mainly where properties of the data are homogeneous and allow conclusions, which might violate the assured k-anonymity. Regarding trajectories, location l-diversity is required as introduced in \citeauthor{Beresford2003} \cite{Beresford2003}. As an enhancement of l-diversity, t-closeness \cite{Li2007} is used on datasets where the distribution of attribute values allows conclusions to identities.

These measurements are typically applied when medical records are published or in regards to Location Based Services (LBS), which require personalized location information. As LBS are usually working with GPS coordinates and trajectories, the raw data is similar to the information. But there is a difference in quality and frequency. LBS usually work with the assumption that a user is willingly providing her location as precise as possible and/or performing measurements of the location with a high frequency. While work has been done on interpolating real trajectories from purposefully obfuscated ones \cite{Naghizade2015}, the data used is limited to one service and focusing on the k of k-anonymity instead on user identification.

The work of \citeauthor{Abul2008} \cite{Abul2008} applies k-anonymity on spatio-temporal objects introducing the $(k, \delta)$-anonymity. The trajectories of a user are extended by the uncertainty of the location measurement $\delta$. The authors claim that a series of trajectories and locations can be modeled as a series of cylinders, or a tube. k-anonymity is granted when $k-1$ additional elements of the set can fit into a tube. The proposed method uses outlier detection and other forms of suppression in combination with space transformation of a maximum of $\delta/2$ while $\delta$ defining the circumference of the tube remains unchanged. The paper proposes a heuristic that succeeds to find anonymity sets as the problem is NP-hard.

The notion of $(k, \delta)$-anonymity is also discussed in \citeauthor{Trujillo-Rasua2013} \cite{Trujillo-Rasua2013}. The authors come to the conclusion
 that existing methods to create $(k, \delta)$-anonymity as developed in \citeauthor{Abul2008} are not sufficient if $\delta > 0$. By defining every location in a spatio-temporal trajectory as a quasi-identifier and assuming that a potential adversary has knowledge about one sub trajectory they show that the probability to correctly identify a series of trajectories is larger than $1/k$ thus violating the $k$-anonymity. This work will show that it is indeed possible to identify users with high probability by only knowing a sub trajectory.
