%!TEX root = ./00_main.tex

\chapter[Additional Research Opportunities]{Additional Research Opportunities}
\label{sec:additional}

There are many additional research opportunities that could follow on this work, in a wide area of topics. Some of those areas might include:

\begin{itemize}
  \item Utilizing this research with addition sources of trajectories to further validate the work in it. A prime target for this would be using twitter and Instagram data to link users across multiple platforms.
  \item Additional research into frequent transitions. This research showed the sequence of locations a user visits can add additional uniqueness, allowing users to be identified more easily, however there is likely room for improvement. The transitions added additions dimensionality to an problem, which was already extremely dimensional, and was not a primary focus of this research.
  \item Dimension reduction, or some form of principle component analysis, could provide significant improvements to this research. The dimensionality of the data will likely cause issues when scaling up to a larger area of interest, and could become a hindrance in very large areas.
  \item A tiered approach could be useful for larger areas of interest, to include world wide analysis. One might be able to use a very coarse grid at a world scale, in order to group users together into smaller areas. Then they could keep drilling down into smaller and smaller areas. At each level, there may be a set of unique users, which are identifiable, and the rest can be drilled down on at a finer scale.
\end{itemize}
